\documentclass[12pt]{exam}

\usepackage[spanish]{babel}
\usepackage[utf8]{inputenc}
%\usepackage[margin=1in]{geometry} \marginsize{.5in}{.5in}{.75in}{.75in}
\usepackage{geometry}
\geometry{
	letterpaper,
	total={170mm,257mm},
	left=15mm,
	top=30mm,
}
\usepackage{pgf,pgfarrows,latexsym,amssymb,amsmath,amsthm,enumerate,array}
\usepackage{multicol}
\usepackage{multirow}
\usepackage{graphicx}
\usepackage{enumerate}
\usepackage{venndiagram}
\usepackage{tikz}

%\usepackage{mathptmx}
%\usepackage{ragged2e}


\newcommand{\clase}{Precálculo}
\newcommand{\examnum}{SEGUNDO EXAMEN PARCIAL}
\newcommand{\examdate}{2 de Abril  de 2020}
\newcommand{\timelimit}{120 Minutos}

\newcommand{\AND}{\wedge}
\newcommand{\OR}{\vee}
\newcommand{\fA}{{\mathfrak{A}}} % Gothic letters, if you want to use them
\newcommand{\cA}{{\mathcal{A}}} % Cal letters, if you want to use them

\renewcommand{\Re}{\operatorname{Re}}
\renewcommand{\Im}{\operatorname{Im}}
\renewcommand{\phi}{\varphi}
\newcommand{\eps}{\varepsilon}
\newcommand{\eqdef}{\coloneqq}
\hyphenation{razona-miento}
\hyphenation{pro-blema}
\hyphenation{estu-diante}
\DeclareMathOperator{\imagunit}{i}
\DeclareMathOperator{\enumber}{e}
\DeclareMathOperator{\Image}{im}
\DeclareMathOperator{\Range}{\mathcal{R}}
\DeclareMathOperator{\sign}{sgn}
\DeclareMathOperator{\closure}{clos}
\DeclareMathOperator{\supp}{supp}
\DeclareMathOperator{\interior}{int}
\DeclareMathOperator{\conv}{conv}
\DeclareMathOperator{\dist}{dist}
\DeclareMathOperator{\SimpleMeasurables}{\mathcal{SM}}
\DeclareMathOperator{\Measurables}{\mathcal{M}}

\newcommand{\matr}[2]{\left(\begin{array}{#1}#2\end{array}\right)}
\newcommand{\detmatr}[2]{\left|\begin{array}{#1}#2\end{array}\right|}

\newcommand{\uniformlyto}[1]{\xLongrightarrow{#1}}
\newcommand{\pointwiseto}[1]{\xlongrightarrow{#1}}
\newcommand{\aeto}[1]{\xlongrightarrow{#1\text{-c.t.p.}}}
\newcommand{\muto}[1]{\xrightharpoondown{#1}}
\newcommand{\aeeq}[1]{\xlongequal{#1\text{-c.t.p.}}}
\newcommand{\auniformlyto}[1]{\xLongrightarrow{#1\text{-c.u.}}}

\newcommand{\com}{ \lq \lq }
\newcommand{\comB}[1]{ \lq \lq #1\rq \rq} %%%% texto entre ""

\definecolor{lightgray}{rgb}{0.8,0.8,0.8}
\definecolor{lightgreen}{rgb}{0.8,1,0.8}
\definecolor{lightblue}{rgb}{0.8,0.8,1}
\newcommand{\h}{1.732}
\newcommand{\hhalf}{0.866}
\newcommand{\Bx}{2}
\newcommand{\By}{0.8}
\newcommand{\Cy}{-3}
\newcommand{\titley}{3}

\pagestyle{head}
\firstpageheader{\Large\sc\clase}{}{\sc\examnum}
\runningheader{\sc\clase}{}{\sc\examnum\ - Pagina \thepage\ de \numpages}
\runningheadrule

\parindent=0mm

\begin{document}
	\vspace{1cm}
	\begin{tabular}{ll}
		\multirow{5}{*}{\includegraphics[scale=0.24]{Sabana1.png}}
		& \large\hspace{0.2cm}Nombre: \makebox[3.8in]{\hrulefill}\vspace{0.2cm}\\
		& \large\hspace{0.2cm}Fecha: \textbf{\examdate} \vspace{0.2cm}\\
		& \large\hspace{0.2cm}21107-Grupo 15    \vspace{0.2cm}\\
		& \large\hspace{0.2cm}Profesor:  HARVEY HERNÁNDEZ YOMAYUSA
	\end{tabular}\\
	\rule[2ex]{\textwidth}{1pt} 
	
	\begin{itemize}
		\scriptsize{\item \textbf{No se permite el uso de elementos electrónicos, smartwatches, etc. La presencia de cualquier equipo de comunicación en el entorno del examen es considerado intento de fraude. Éste o cualquier otro intento de fraude implica cero en el examen y se reporta a la facultad.}
			\item \textbf{La comprensión del examen es parte de la evaluación, por tanto, no se responden preguntas durante el desarrollo de éste.} 
			\item \textbf{Se evalúa procedimiento y/o justificación, por tanto, sea claro y ordenado. Respuesta sin justificar no será válida.}
			% \item \textbf{Este examen contiene \numpages\ páginas y \numquestions\ preguntas.  El Total de puntos en esta prueba es de \numpoints}
		}	
		
	\end{itemize} 
	
	%\texttt{\noindent}
	\rule[2ex]{\textwidth}{1pt}

\begin{questions}

\question[12] Si se tiene la frase \comB{\textbf{El precio del petróleo no aumentará,  si no hay conflictos  en Medio Oriente o Estados Unidos no cambia las políticas económicas}}.


\begin{parts}
	
	\part[4]\framebox[1.5cm][c]{R1-B1}  Identifique y escriba cada una de las proposiciones simples que conforman la frase.\\
	
	\part[4] \framebox[1.5cm][c]{R1-B1} Escribir la frase en lenguaje formal (simbólico)\\
	\part[4] \framebox[1.5cm][c]{R3-B3} Construir la tabla de verdad.
\end{parts}
	
\question[8]  En los siguientes ejercicios demostrar que una proposición es consecuencia lógica de las premisas dadas completando el esquema.
\begin{parts}
	\part[4] \framebox[1.5cm]{R3-C1} Identifique y use las reglas de inferencia que se aplican en el siguiente argumento para obtener una conclusión válida.\\
	\begin{center}
		\begin{minipage}{5cm}
			\begin{itemize}
		\item[(P1)] $ A \implies \lnot W  $  
		\item[(P2)] $ \lnot S \implies R $   
		\item[(P3)] $ \lnot M \implies K $
		\item[(P4)] $ N \implies A  $ 
		\item[(P5)] $ K \implies \lnot S $
		\item[(P6)] $ R \implies N $
		\item[(P7)] $ W $
		\end{itemize}
	\end{minipage}
	\end{center}
\part[4] \framebox[1.5cm]{R3-C3} Teniendo en cuenta el procedimiento realizado en el punto anterior, si se pidiera plantear el ejercicio como demostración, ¿Cómo se formularía el enunciado de la demostración?\\

\end{parts}


\newpage
\question[14]  Se tiene el siguiente enunciado

\comB{\textbf{Si el ratón se come el queso, entonces el gato atrapa al ratón. Pero el gato no atrapa al ratón. Por tanto, el ratón no se come el queso}}.


\begin{parts}
	\part[4]\framebox[1.5cm][c]{R1-B1} Represente simbólicamente las proposiciones y las premisas planteadas
	
	\part[4] \framebox[1.5cm][c]{R3-B3} ¿Qué conclusión se puede inferir a partir de las premisas?,Escríbala primero en lenguaje natural y luego  simbolícela.
	
	\part[4] \framebox[1.5cm][c]{R3-B2} Realice el procedimiento necesario para validar el argumento.
	
	\part[2] \framebox[1.5cm][c]{R3-B3} A partir de los procedimientos anteriores que puede concluir, ¿El argumento es válido si o nó?. 
	
\end{parts}


\question[8] Aplicando las Leyes de Morgan escriba las proposiciones equivalentes en su forma de negacion:
\begin{parts}
	\part[4] \framebox[1.5cm][c]{R3-B2} Este semestre vemos 7 materias y todas están complicadas.
	 
	%	\textcolor{red}{R./No es cierto que, Este semestre vemos 7 materias o no todas están complicadas. }
	\part [4] \framebox[1.5cm][c]{R3-B2} El trabajo estaba difícil  o yo estaba confundido.\\
 
	%	\textcolor{red}{R./ No es cierto que, el trabajo no estaba difícil y  yo no estaba confundido. }
\end{parts}

\question[8] Dada la frase: \comB{\textbf{No todos los países tienen un buen sistema de salud}}.
		\begin{parts}
		\part[4] \framebox[1.5cm][c]{R1-B1} Simbolice  la frase\\
		\part[4] \framebox[1.5cm][c]{R3-C2} Escriba la negación de la frase utilizando la negación del cuantificador, de forma verbal.\\
	 
		
	
		
		%\noindent\rule{15cm}{0.4pt}\\
	\end{parts}





\end{questions}

\end{document}